\section{Conclusioni} Il lavoro svolto ha portato allo sviluppo di un sistema di sintesi automatica in grado di generare riassunti coerenti a partire da recensioni di cibo. Il sistema è stato progettato per garantire facilità di estensione e adattabilità a diverse tipologie di task, aumentando così la sua versatilità e potenziale applicativo in contesti variegati.\\
Nel corso del progetto sono stati implementati diversi modelli di sintesi automatica, tra cui Seq2SeqLSTM, Seq2SeqBiLSTM, Seq2Seq3BiLSTM e Seq2SeqLSTMGlove. 
I risultati sono stati valutati attraverso numerose metriche, quali ROUGE, Word Error Rate (WER), Cosine Similarity, BERTScore e \texttt{MyEvaluation}. In particolare, 
la BERTScore si è rivelata la metrica più rappresentativa della qualità dei riassunti, in quanto considera il contesto semantico e la similarità delle parole.\\

Inoltre, la metrica \texttt{MyEvaluation} ha fornito una valutazione più dettagliata della qualità dei riassunti, integrando vari fattori e assegnando a ciascuno 
un peso in base alla sua rilevanza. I risultati sperimentali indicano che il modello \textbf{Seq2SeqBiLSTM} ha ottenuto le prestazioni migliori su quasi tutte le metriche, 
suggerendo una superiore capacità di catturare le similarità semantiche tra i riassunti generati e quelli di riferimento.\\

È opportuno sottolineare che i risultati sono stati influenzati dalla scelta del dataset e dalle specifiche caratteristiche dei dati, 
inclusi gli aspetti relativi al preprocessing. Tali elementi evidenziano l'importanza di un'attenta selezione e preparazione dei dati per 
ottimizzare le prestazioni dei modelli di sintesi automatica.