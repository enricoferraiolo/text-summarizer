\subsection{Seq2SeqLSTMGloVe}
La classe \texttt{Seq2SeqLSTMGloVe} implementa un'architettura simile al modello Seq2SeqLSTM, utilizzando i vettori di embedding GloVe preaddestrati per la rappresentazione delle parole.\\
Più precisamente vengono scaricati e utilizzati i vettori di embedding GloVe preaddestrati da \href{https://nlp.stanford.edu/projects/glove/}{Stanford NLP Group} da 100 dimensioni, anche se la classe consente di scambiare facilmente i vettori con quelli di dimensione diversa.\\

\training{Adam}{50}
\risultati{DA INSERIRE}{DA INSERIRE}{DA INSERIRE}{DA INSERIRE}
Possiamo verifcare l'andamento delle loss durante l'addestramento nella figura \ref{fig:seq2seqlstmglove_loss_plot}.
\begin{figure}[H]
    \centering
    \includegraphics[width=0.75\textwidth]{media/Seq2SeqLSTMGloVe_originale_lossplot.png}
    \caption{Andamento delle loss durante l'addestramento}
    \label{fig:seq2seqlstmglove_loss_plot}
\end{figure}
