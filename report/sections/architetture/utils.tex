\newcommand{\risultati}[4]{
\subsubsection{Risultati}
Questo modello ha ottenuto i seguenti risultati al termine dell'addestramento:
\begin{itemize}
    \item \textbf{Loss}: #1
    \item \textbf{Validation loss}: #2
    \item \textbf{Accuracy}: #3
    \item \textbf{Validation Accuracy}: #4
\end{itemize}
}

\newcommand{\training}[8]{
\subsubsection{Training}
L'addestramento del modello è stato effettuato attraverso la seguente configurazione:
\begin{itemize}
    \item Ottimizzatore: #1
    \item Learning rate: #2
    \item Embedding dimension: #3
    \item Latent dimension: #4
    \item Decoder dropout: #5
    \item Decoder recurrent dropout: #6
    \item Encoder dropout: #7
    \item Encoder recurrent dropout: #8
    \trainingcontinued
}

\newcommand{\trainingcontinued}[2]{
    \item Batch size: #1
    \item Epochs: #2
\end{itemize}

\lossimg
}

\newcommand{\lossimg}[2]{
Possiamo verificare l'andamento delle loss durante l'addestramento nella figura \ref{fig:#2}.

\begin{figure}[H]
    \centering
    \includegraphics[width=0.75\textwidth]{media/#1_best_lossplot.png}
    \caption{Andamento delle loss durante l'addestramento}
    \label{fig:#2}

\end{figure}
}

\newcommand{\architecture}[3]{
    \subsubsection{Architettura}
    L'architettura utilizzata è la seguente:
    \archimg{#1}{#2}{#3}
}

\newcommand{\archimg}[3]{
\begin{figure}[H]
    \centering
    \includegraphics[width=0.75\textwidth]{media/#1_best_architecture.png}
    \caption{Architettura del modello #2}
    \label{fig:#3}
\end{figure}
}