\definecolor{lightgray}{gray}{0.95}

\newcommand{\risultati}[4]{
  \subsubsection{Results}
  This model achieved the following results at the end of training:
  \begin{table}[H]
  \centering
  \rowcolors{2}{lightgray}{white}
  \begin{tabular}{@{} >{\bfseries}l c @{}}
    \toprule
    \textbf{Metric}    & \textbf{Value} \\
    \midrule
    Loss                & #1              \\
    Validation Loss     & #2              \\
    Accuracy            & #3              \\
    Validation Accuracy & #4              \\
    \bottomrule
  \end{tabular}
  \caption{Results of the model at the end of training}
  \label{tab:risultati}
  \end{table}
}

\newcommand{\training}[8]{
  \subsubsection{Training}
  The training of the model was carried out using the following configuration:
  \begin{table}[ht]
    \centering
    \rowcolors{2}{lightgray}{white}
    \begin{tabular}{@{} >{\bfseries}l c @{}}
      \toprule
      \textbf{Parameter}        & \textbf{Value} \\
      \midrule
      Optimizer                 & #1              \\
      Learning rate             & #2              \\
      Embedding dimension       & #3              \\
      Latent dimension          & #4              \\
      Decoder dropout           & #5              \\
      Decoder recurrent dropout & #6              \\
      Encoder dropout           & #7              \\
      Encoder recurrent dropout & #8              \\
      \trainingcontinued
      }

      \newcommand{\trainingcontinued}[2]{
      Batch size                & #1              \\
      Epochs                    & #2              \\
      \bottomrule
    \end{tabular}
    \caption{Training configuration}
  \end{table}

  \lossimg
}

\newcommand{\lossimg}[2]{
  We can check the trend of the losses during training in Figure \ref{fig:#2}.

  \begin{figure}[H]
    \centering
    \includegraphics[width=0.75\textwidth]{media/#1_best_lossplot.png}
    \caption{Trend of the \textit{loss} and \textit{validation loss} during training}
    \label{fig:#2}

  \end{figure}
}

\newcommand{\architecture}[3]{
  \subsubsection{Architecture}
  The architecture used is as follows:
  \archimg{#1}{#2}{#3}
}

\newcommand{\archimg}[3]{
  \begin{figure}[H]
    \centering
    \includegraphics[width=0.75\textwidth]{media/#1_best_architecture.png}
    \caption{Architecture of the model #2}
    \label{fig:#3}
  \end{figure}
}

\newcommand{\rougeonesubimg}[3]{
  \begin{subfigure}{#1\textwidth}
    \centering
    \includegraphics[width=\textwidth]{media/#2_best_rouge1_scores.png}
    \caption{ROUGE-1 #2}
    \label{fig:#3}
  \end{subfigure}
}

\newcommand{\rougetwosubimg}[3]{
  \begin{subfigure}{#1\textwidth}
    \centering
    \includegraphics[width=\textwidth]{media/#2_best_rouge2_scores.png}
    \caption{ROUGE-2 #2}
    \label{fig:#3}
  \end{subfigure}
}

\newcommand{\rougelubimg}[3]{
  \begin{subfigure}{#1\textwidth}
    \centering
    \includegraphics[width=\textwidth]{media/#2_best_rougeL_scores.png}
    \caption{ROUGE-L #2}
    \label{fig:#3}
  \end{subfigure}
}

\newcommand{\wersubimg}[3]{
  \begin{subfigure}{#1\textwidth}
    \centering
    \includegraphics[width=\textwidth]{media/#2_best_wer_scores.png}
    \caption{WER #2}
    \label{fig:#3}
  \end{subfigure}
}

\newcommand{\cssubimg}[3]{
  \begin{subfigure}{#1\textwidth}
    \centering
    \includegraphics[width=\textwidth]{media/#2_best_cosine_similarity_scores.png}
    \caption{Cosine similarity #2}
    \label{fig:#3}
  \end{subfigure}
}

\newcommand{\bertsubimg}[3]{
  \begin{subfigure}{#1\textwidth}
    \centering
    \includegraphics[width=\textwidth]{media/#2_best_bert_scores.png}
    \caption{BERT score #2}
    \label{fig:#3}
  \end{subfigure}
}

\newcommand{\myevalsubimg}[3]{
  \begin{subfigure}{#1\textwidth}
    \centering
    \includegraphics[width=\textwidth]{media/#2_best_myevaluation_scores.png}
    \caption{My Evaluation #2}
    \label{fig:#3}
  \end{subfigure}
}