\newcommand{\risultati}[4]{
\subsubsection{Risultati}
Questo modello ha ottenuto i seguenti risultati al termine dell'addestramento:
\begin{itemize}
    \item \textbf{Loss}: #1
    \item \textbf{Validation loss}: #2
    \item \textbf{Accuracy}: #3
    \item \textbf{Validation Accuracy}: #4
\end{itemize}
}

\definecolor{lightgray}{gray}{0.95}

\newcommand{\training}[8]{
\subsubsection{Training}
L'addestramento del modello è stato effettuato attraverso la seguente configurazione:
\begin{table}[ht]
    \centering
    \rowcolors{2}{lightgray}{white}
    \begin{tabular}{@{} >{\bfseries}l c @{}}
      \toprule
      \textbf{Parametro} & \textbf{Valore} \\
      \midrule
      Ottimizzatore            & #1 \\
      Learning rate            & #2 \\
      Embedding dimension      & #3 \\
      Latent dimension         & #4 \\
      Decoder dropout          & #5 \\
      Decoder recurrent dropout& #6 \\
      Encoder dropout          & #7 \\
      Encoder recurrent dropout& #8 \\
      \trainingcontinued
      }
      
      \newcommand{\trainingcontinued}[2]{
        Batch size               & #1 \\
        Epochs                   & #2\\
        \bottomrule
      \end{tabular}
      \caption{Configurazione dell'addestramento del modello}
  \end{table}

\lossimg
}

\newcommand{\lossimg}[2]{
Possiamo verificare l'andamento delle loss durante l'addestramento nella figura \ref{fig:#2}.

\begin{figure}[H]
    \centering
    \includegraphics[width=0.75\textwidth]{media/#1_best_lossplot.png}
    \caption{Andamento della \textit{loss} e \textit{validation loss} durante l'addestramento}
    \label{fig:#2}

\end{figure}
}

\newcommand{\architecture}[3]{
    \subsubsection{Architettura}
    L'architettura utilizzata è la seguente:
    \archimg{#1}{#2}{#3}
}

\newcommand{\archimg}[3]{
\begin{figure}[H]
    \centering
    \includegraphics[width=0.75\textwidth]{media/#1_best_architecture.png}
    \caption{Architettura del modello #2}
    \label{fig:#3}
\end{figure}
}

\newcommand{\rougeonesubimg}[3]{
  \begin{subfigure}{#1\textwidth}
    \centering
    \includegraphics[width=\textwidth]{media/#2_best_rouge1_scores.png}
    \caption{ROUGE-1 #2}
    \label{fig:#3}
  \end{subfigure}
}

\newcommand{\rougetwosubimg}[3]{
  \begin{subfigure}{#1\textwidth}
    \centering
    \includegraphics[width=\textwidth]{media/#2_best_rouge2_scores.png}
    \caption{ROUGE-2 #2}
    \label{fig:#3}
  \end{subfigure}
}

\newcommand{\rougelubimg}[3]{
  \begin{subfigure}{#1\textwidth}
    \centering
    \includegraphics[width=\textwidth]{media/#2_best_rougeL_scores.png}
    \caption{ROUGE-L #2}
    \label{fig:#3}
  \end{subfigure}
}

\newcommand{\wersubimg}[3]{
  \begin{subfigure}{#1\textwidth}
    \centering
    \includegraphics[width=\textwidth]{media/#2_best_wer_scores.png}
    \caption{WER #2}
    \label{fig:#3}
  \end{subfigure}
}

\newcommand{\cssubimg}[3]{
  \begin{subfigure}{#1\textwidth}
    \centering
    \includegraphics[width=\textwidth]{media/#2_best_cosine_similarity_scores.png}
    \caption{Cosine similarity #2}
    \label{fig:#3}
    \end{subfigure}
}

\newcommand{\bertsubimg}[3]{
  \begin{subfigure}{#1\textwidth}
    \centering
    \includegraphics[width=\textwidth]{media/#2_best_bert_scores.png}
    \caption{BERTScore #2}
    \label{fig:#3}
    \end{subfigure}
}

\newcommand{\myevalsubimg}[3]{
  \begin{subfigure}{#1\textwidth}
    \centering
    \includegraphics[width=\textwidth]{media/#2_best_myevaluation_scores.png}
    \caption{My Evaluation #2}
    \label{fig:#3}
    \end{subfigure}
}