\newcommand{\risultati}[4]{
\subsubsection{Risultati}
Questo modello ha ottenuto i seguenti risultati al termine dell'addestramento:
\begin{itemize}
    \item \textbf{Loss}: #1
    \item \textbf{Validation loss}: #2
    \item \textbf{Accuracy}: #3
    \item \textbf{Validation Accuracy}: #4
\end{itemize}
}

\newcommand{\training}[2]{
\subsubsection{Training}
L'addestramento del modello è stato effettuato con ottimizzatore #1 e #2 epoche con early stopping.\\
}


\section{Architettura del Modello}
L'implementazione dei modelli è stata effettuata attraverso una classe astratta \texttt{BaseModel} e una classe derivata \texttt{   LSTM}.\\
Questo permette di definire un'interfaccia comune per tutti i modelli di summarization e di estendere facilmente l'architettura in futuro.\\

\subsection{Classe Base Astratta}
La classe \texttt{BaseModel} fornisce l'interfaccia base per tutti i modelli di summarization:
\begin{itemize}
    \item Metodi astratti per costruire encoder e decoder.
    \item Funzionalità per il salvataggio, caricamento e inferenza del modello.
    \item Conversione tra sequenze e testo tramite i tokenizzatori.
\end{itemize}

\subsection{Training}
L'addestramento dei modelli successivi è stato effettuato utilizzando il dataset preprocessato.\\
Prima di iniziare l'addestramento, il dataset è stato suddiviso in training set e validation set, con una proporzione del 90\% e 10\% rispettivamente.\\
A questo punto ho definito una funzione di early stopping per monitorare la loss sul validation set e fermare l'addestramento quando la loss non diminuisce per un certo numero di epoche, ciò per evitare l'overfitting.\\
Dopodiché sono passato alla fase effettiva di training del modello, utilizzando l'ottimizzatore \texttt{Adam} e la loss function \texttt{Sparse Categorical Crossentropy}.\\

\subsection{Seq2SeqLSTM}
La classe \texttt{Seq2SeqLSTM} implementa l'architettura specifica per il modello di summarization Sequence to Sequence con layer LSTM.

\training{Adam}{0.001}{512}{128}{0.2}{0.2}{0.2}{0.2}{128}{50}{Seq2SeqLSTM}{seq2seqlstm_loss_plot}
\risultati{1.63}{2.02}{0.67}{0.64}
\architecture{Seq2SeqLSTM}{Seq2SeqLSTM}{seq2seqlstm_architecture}

\subsection{Seq2SeqBiLSTM}
La classe \texttt{Seq2SeqBiLSTM} implementa un'architettura simile al modello Seq2SeqLSTM, ma i layer LSTM dell'encoder sono bidirezionali.

\training{Adam}{0.001}{512}{256}{0.2}{0.2}{0.2}{0.2}{128}{50}{Seq2SeqBiLSTM}{seq2seqbilstm_loss_plot}
\risultati{1.35}{1.98}{0.70}{0.65}
\architecture{Seq2SeqBiLSTM}{Seq2SeqBiLSTM}{seq2seqbilstm_architecture}
\subsection{Seq2Seq3BiLSTM}
La classe \texttt{Seq2Seq3BiLSTM} implementa un'architettura simile al modello Seq2SeqBiLSTM, ma con tre layer LSTM bidirezionali nell'encoder.\\
Di seguito, possiamo vedere un diagramma dell'architettura del modello Seq2Seq3BiLSTM nella figura \ref{fig:seq2seq3bilstm_model_architecture}.
\begin{figure}[H]
    \centering
    \includegraphics[width=1\textwidth]{media/Seq2Seq3BiLSTM_image.png}
    \caption{Diagramma dell'architettura del modello Seq2Seq3BiLSTM}
    \label{fig:seq2seq3bilstm_model_architecture}
\end{figure}

\training{Adam}{50}
\risultati{DA INSERIRE}{DA INSERIRE}{DA INSERIRE}{DA INSERIRE}

\begin{figure}[H]
    \centering
    %TODO: Aggiungere immagine traingin loss È DA AGGIORNARE
    \includegraphics[width=0.75\textwidth]{media/Seq2Seq3BiLSTM_originale_lossplot.png}
    \caption{Andamento delle loss durante l'addestramento}
    \label{fig:seq2seq3bilstm_loss_plot}
\end{figure}


\input{sections/architetture/Seq2SeqLSTMGloVe.tex}
\subsection{Seq2SeqLSTMTrasformer}
La classe \texttt{Seq2SeqLSTMTrasformer} implementa un'architettura simile al modello Seq2SeqLSTM, ma inoltre aggiunge \(2\) blocchi Transformer sia nell'encoder che nel decoder.\\

\subsubsection{Encoder}
\begin{itemize}
    \item \textbf{Layer di embedding}
    \item \textbf{Due Blocchi Transformer} con:
        \begin{itemize}
            \item Dimensione latente pari a 300
            \item Dropout del 40\%
        \end{itemize}
    \item \textbf{Tre layer LSTM} con:
    \begin{itemize}
        \item Dimensione latente pari a 300
        \item Dropout del 40\%
        \item Recurrent dropout del 40\%
    \end{itemize}
\end{itemize}

\subsubsection{Decoder}
\begin{itemize}
    \item \textbf{Layer di embedding}
    \item \textbf{Due blocchi Transformer} con:
        \begin{itemize}
            \item Dimensione latente pari a 300
            \item Dropout del 40\%
            \item Recurrent dropout del 20\%
            \item Un MultiHeadAttention da \(8\) Head
            \item Un layer di Dropout del 10\%
            \item Un Add layer
            \item Un layer di Normalizzazione
            \item Tre layer di FeedForward con:
                \begin{itemize}
                    \item Un layer denso con attivazione ReLU e dimensione latente pari a \(512\)
                    \item Un layer denso con dimensione latente pari a \(300\)
                    \item Un layer di Dropout del 10\%
                    \item Un Add layer
                    \item Un layer di Normalizzazione
                \end{itemize}
        \end{itemize}
    \item \textbf{Layer LSTM} con:
        \begin{itemize}
            \item Stessa dimensione latente dell'encoder
            \item Dropout del 40\%
            \item Recurrent dropout del 40\%
        \end{itemize}
    \item \textbf{Layer di attention}
    \item \textbf{Layer denso di output}
\end{itemize}

Di seguito, possiamo vedere un diagramma dell'architettura del modello Seq2SeqLSTMTrasformer nella figura \ref{fig:seq2seqlstmtrasformer_model_architecture}.
\begin{figure}[H]
    \centering
    \includegraphics[width=0.55\textwidth]{media/Seq2SeqLSTMTransformer_image.png}
    \caption{Diagramma dell'architettura del modello Seq2SeqLSTMTrasformer}
    \label{fig:seq2seqlstmtrasformer_model_architecture}
\end{figure}

\training{Adam}{50}
\risultati{0.2015}{0.5034}{0.9576}{0.9269}

Possiamo verificare l'andamento della funzione di loss durante l'addestramento del modello nella figura \ref{fig:seq2seqlstmtrasformer_loss}.
\begin{figure}[H]
    \centering
    \includegraphics[width=0.65\textwidth]{media/Seq2SeqLSTMTransformer_originale_lossplot.png}
    \caption{Andamento della funzione di loss durante l'addestramento del modello Seq2SeqLSTMTrasformer}
    \label{fig:seq2seqlstmtrasformer_loss}
\end{figure}
