\section{Introduzione}
L'obiettivo principale del progetto è generare riassunti concisi e significativi a partire da recensioni di prodotti più lunghe, mantenendo il significato del testo originale.\\
Il progetto si articola nelle seguenti fasi:
\begin{itemize}
    \item \textbf{Raccolta e preparazione dei dati:} selezione e pre-elaborazione di un dataset di recensioni di prodotti, con particolare attenzione alla pulizia e alla normalizzazione del testo.
    \item \textbf{Progettazione e implementazione di architetture di reti neurali:} studio e sviluppo di modelli basati su meccanismi di attenzione per la sintesi testuale.
    \item \textbf{Addestramento e inferenza:} realizzazione di pipeline per l'addestramento dei modelli e per l'esecuzione delle operazioni di sintesi su nuovi testi.
    \item \textbf{Valutazione sperimentale:} analisi comparativa delle prestazioni dei modelli mediante metriche standardizzate, al fine di identificare le soluzioni ottimali.
\end{itemize}
Questo documento vuole illustrare le scelte progettuali e le metodologie adottate per la realizzazione del progetto, nonché i risultati sperimentali ottenuti.

\subsection{Struttura del progetto}
Il progetto è strutturato in diverse sezioni, ognuna delle quali affronta un aspetto specifico del lavoro svolto.\\
La cartella principale contiene i file e le cartelle necessarie per l'esecuzione del progetto:
\begin{itemize}
    \item \textbf{architectures}: cartella contenente le implementazioni delle architetture dei modelli di sintesi automatica;
    \item \textbf{report}: cartella contenente il report finale del progetto;
    \item \textbf{results}: cartella contenente i risultati ottenuti durante l'addestramento e la valutazione dei modelli;
    \item \textbf{\texttt{requirements.txt}}: file contenente le librerie necessarie per l'esecuzione del progetto;
    \item \textbf{\texttt{text\_summarizer\_training.ipynb}}: notebook Python contenente il codice per l'addestramento e la valutazione dei modelli;
    \item \textbf{\texttt{text\_summarizer\_inference.ipynb}}: notebook Python contenente il codice per l'inferenza dei modelli.
\end{itemize}
